\documentclass{article}

\usepackage{fancyhdr} 
\usepackage{lastpage} 
\usepackage{extramarks} 
\usepackage[usenames,dvipsnames]{color}
\usepackage{graphicx} 
\usepackage{listings} 
\usepackage{courier} 
\usepackage{multirow}
\usepackage{hyperref}

\topmargin=-0.45in
\evensidemargin=0in
\oddsidemargin=0in
\textwidth=6.5in
\textheight=9.0in
\headsep=0.25in

\linespread{1.1} 

\definecolor{MyDarkGreen}{rgb}{0.0,0.4,0.0} 
\lstloadlanguages{c} 
\lstset{language=[sharp]c,frame=single,
        basicstyle=\small\ttfamily, 
        keywordstyle=[1]\color{Blue}\bf,
        keywordstyle=[2]\color{Purple}, 
        keywordstyle=[3]\color{Blue}\underbar, 
        identifierstyle=,                           
        commentstyle=\usefont{T1}{pcr}{m}{sl}\color{MyDarkGreen}\small, 
        stringstyle=\color{Purple}, 
        showstringspaces=false, 
        tabsize=5, 
                morekeywords={rand},
                morekeywords=[2]{on, off, interp},
        morekeywords=[3]{test},
        morecomment=[l][\color{Blue}]{...},
        numbers=left, 
        firstnumber=1,        numberstyle=\tiny\color{Blue}, 
        stepnumber=5 
}
\newcommand{\horrule}[1]{\rule{\linewidth}{#1}}

\newcommand{\perlscript}[2]{
\begin{itemize}
\item[]\lstinputlisting[caption=#2,label=#1]{#1.cs}
\end{itemize}
}

\begin{document}


\begin{center}
        \horrule{0.5pt}
        \huge{Hoja de trabajo \#1} \\
        \large{Gabriel Chavarria, 20181386, chavarria181386@unis.edu.gt} \\
\large{25 de julio 2018}\\        
        \horrule{1pt}
\end{center}


\section*{Ejercicio \#2 Abstracción}
\begin{enumerate}
        \item{ Conjunto de nodos: $\left\lbrace 1,2,3,4,5,6\right\rbrace$}
        
       
\end{enumerate}
    
\section*{Ejercicio \#3}
\begin{enumerate}
        \item{ Una estructura de camino}
        \item{Modelo algebraico que siga una estructura de camino, donde este permita que un mismo numero pueda salir más de  una vez, y también logre manejar el azar que se requiere.  }
        \item{Nos aseguramos que no se repita el proceso, se evita que se cree un ciclo.}
                
\end{enumerate}

\end{document}
       
        


